\ssr{ВВЕДЕНИЕ}

Динамические сцены с природными объектами, такими как поля подсолнухов, находят широкое применение в компьютерных играх и виртуальных средах. Они позволяют создавать атмосферу и усиливать погружение пользователя в виртуальный мир. Однако реализация таких сцен требует решения ряда задач, включая моделирование движения объектов под воздействием внешних факторов (например, ветра) и учёт освещения в зависимости от времени суток.

Целью данной работы является разработка программного обеспечения для построения трёхмерной динамической сцены, состоящей из поля подсолнухов, которые качаются под воздействием ветра, и источника света, положение которого зависит от заданного пользователем времени суток.

\vspace{0.5cm}
\textbf {Для достижения поставленной цели необходимо решить следующие задачи:}
\begin{enumerate}
	\item проанализировать предметую область;
    \item проанализировать существующие подходы и алгоритмы для моделирования и визуализации природных объектов;
    \item выбрать средства реализации программного обеспечения;
    \item создать программное обеспечение;
    \item создать систему освещения, которая будет изменяться в зависимости от времени суток, заданного пользователем;
    \item разработать пользовательский интерфейс для настройки параметров сцены;
    \item провести тестирование и исследование разработанного программного обеспечения для оценки его производительности.
\end{enumerate}


\clearpage
