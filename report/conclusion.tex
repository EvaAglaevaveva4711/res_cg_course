\ssr{ЗАКЛЮЧЕНИЕ}

\vspace{0.5cm}
В ходе работы были выполнены следующие задачи:
\begin{enumerate}
\item проанализирована предметная область;
\item изучены существующие подходы и алгоритмы для моделирования и визуализации природных объектов;
\item выбраны средства реализации программного обеспечения;
\item создано программное обеспечение, которое моделирует трёхмерную динамическую сцену с подсолнухами, реагирующими на ветер;
\item реализована система освещения, изменяющаяся в зависимости от времени суток;
\item разработан пользовательский интерфейс, позволяющий настраивать параметры сцены;
\item проведено тестирование и исследование разработанного программного обеспечения, результаты которого подтвердили его высокую производительность и стабильность при различных настройках.
\end{enumerate}

\vspace{0.5cm}
В результате проведённого исследования была разработана трёхмерная динамическая сцена, демонстрирующая реалистичное поведение подсолнухов под воздействием ветра и адаптивное освещение. Это программное обеспечение может быть использовано в различных приложениях, включая компьютерные игры и виртуальные симуляции. Полученные результаты подтвердили целесообразность выбранных подходов и методов, а также их практическую применимость в области компьютерной графики и анимации.

